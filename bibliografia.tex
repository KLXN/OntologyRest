\begin{thebibliography}{1}

\bibitem{1}
Erik Sundvall, Mikael Nystrom, Daniel Karlsson, Martin Eneling, Rong Chen, Hakan Orman: Applying REST architecture to archetype-based electronic health record systems

\bibitem{2}
Wiesław Wajs, Krzysztof Rączka, Paweł Stoch, Piotr Kruczek: Integration Platform As Central Service Of Data Replication In Distributed Medical System

\bibitem{3}
Bartosz Jędrzejec: Pozyskiwanie wiedzy z dużych zbiorów danych z zastosowaniem adaptacyjnych procedur generowania zapytań, AGH Kraków 2008

\bibitem{4}
Dortje Loper, Meike Klettke, Livio Bruder, Andreas Heuer: Enabling flexible integration of healthcare information using entity-attribute-value storage model

\bibitem{5}
IHE cross-enterprise document sharing for imaging: interoperability testing software - Rita Noumeir, Berube Renaud - Springer Link database

\bibitem{6}
Integrating Clinical Trial Imaging Data Resources Using Service-Oriented Architecture and Grid Computing - Stefan Baumann El-Ghatta, Thierry Cladé, Joshua C. Snyder - Springer Link database

\bibitem{7}
Welcome to Health Information Science and Systems - Yanchun Zhang
LabKey Server: An open source platform for scientific data integration, analysis and collaboration - Elizabeth K Nelson, Britt Piehler - Springer Link database

\bibitem{8}
A database application for pre-processing, storage and comparison of mass spectra derived from patients and controls - Mark K Titulaer, Ivar Siccama - Springer Link database

\bibitem{9}
PASSIM – an open source software system for managing information in biomedical studies - Juris Viksna, Edgars Celms - Springer Link database

\bibitem{10}
The HL7 Clinical Document Architecture - Robert H Dolin, Liora Alschuler, Calvin Beebe; Journal of the American Medical Informatics Association Volume 8 Number 6

\bibitem{11}
The HL7 Clinical Document Architecture, Release 2 - Robert H Dolin, Liora Alschuler, Calvin Beebe; Journal of the American Medical Informatics Association Volume 13 Number 1

\bibitem{12}
HL7 Document Patient Record Architecture: An XML Document Architecture Based on a Shared Information Model - Robert H. Dolin, MD, Liora Alschuler, Fred Behlen, PhD, Paul V. Biron, MLIS, Sandy Boyer, RPh; Dan Essin, MD, Lloyd Harding, Tom Lincoln, MD, John E. Mattison, MD, Wes Rishel, Rachael Sokolowski, John Spinosa, MD, PhD, Jason P. Williams, MS

\bibitem{13}
HL7 Version 3—An object-oriented methodology for collaborative standards development - George W Beeler
A health-care data model based on the HL7 Reference Information Model - Eggebraaten, T.J.
Development of a Clinical Data Warehouse for Hospital Infection Control - Mary F Wisniewski, Piotr Kieszkowski, Brandon M Zagorski, 

\bibitem{14}
The HL7 Reference Information Model Under Scrutiny - Gunther Schadow , Charles N. Mead, D. Mead Walker
Biomedical Engineering (Chapter 19 - Toward Multi-Service Electronic Medical Records Structure) - Bilal I. Alqudah, Suku Nair

\bibitem{15}
Introduction to: HL7 Reference Information Model (RIM) - Health Level Seven International

\bibitem{16}
HL7 RIM: An Incoherent Standard - Barry Smith, Werner Ceusters

\bibitem{17}
Redundancy in electronic health records corpora: analysis, impact on text mining performance and mitigation strategies - Raphael Cohen, Michael Elhadad, Noemie Elhadad

\bibitem{18}
Initial experience with asynchronous transfer mode for use in medical imaging network - Minh Dovan, Louis M. Humphrey, Geri Cox, Carl E. Ravin

\bibitem{19}
A database application for pre-processing, storage and comparison of mass spectra derived  from patients and controls - Mark T Titulaer, Ivar Siccama, Lennard J Dekker

\bibitem{20}
Adding HL7 version 3 data types to PostgreSQL - Yeb Havinga, Willem Dijkstra, Ander de Keijzer

\bibitem{21}
Clinical data integration of distributed data sources using Health Level Seven (HL7) v3-RIM mapping - Teeradache Viangteeravat, Matthew N Anyanwu, Venkateswara Ra Nagisetty, Emin Kuscu, Mark Eijiro Sakauye, Duojiao Wu

\bibitem{22}
Electronic Medical Records vs. Electronic Health Records: Yes, There Is a Difference - Dave Garets and Mike Davis

\bibitem{23}
The “New” America Electronic Medical Record (EMR)—Design Criteria and Challenge - Ralph Grams

\bibitem{24}
Security of Medical Data Transfer and Storage in Internet. Cryptography, Antiviral Security and Electronic Signature Problems, which Must Be Solved in Nearest Future in Practical Context - Piotr Kasztelowicz, Marek Czubenko, Iwona Zięba

\bibitem{25}
Barriers to implement Electronic Health Records (EHRs) - Sima Ajami, Razieh Arab-Chadegani

\end{thebibliography}