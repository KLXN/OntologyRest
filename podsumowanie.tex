\section{Podsumowanie}
\label{cha:podsumowanie}

Zaproponowana przez nas implementacja reprezentacji oraz transferu danych medycznych posiada wiele cech pożądanych przy projektowaniu tego typu systemów. Rozwiązanie to cechuje się bardzo dużą skalowalnoscią zarówno od strony wielkosci samego systemu jak i możliwosci rozbudowy przechowywanych w nim danych. Zastosowanie architektury typu SOA przy budowie systemu medyczneo pozwala na łatwą, modułową rozbudowę takiego systemu oraz ułatwioną integrację z innymi systemami medycznymi, zupełnie niezależnymi od siebie. Idąc dalej można skorzystać również z możliwosci oferowanych przez narzędzia jeszcze wyższych rzędów, takie jak szyny biznesowe, dzięki którym integracja mogłaby być jeszcze skuteczniejsza, nie naruszając jednoczesnie architektur oraz implementacji poszczególnych systemów.

Poprzez prezentacje przypadku użycia pokazaliśmy jak łatwo można tworzyć aplikacje klienckie do przygotowanej przez nas platformy. Dodawanie obsługi nowych przypadków użycia lub dodawanie nowych aktorów (np. pacjentów, recepcjonistek) nie powinno stanowić problemów. Proces ten proponujemy przeprowadzać analogicznie do przygotowanego przez nas przykładu: zaczynając od ustalenia przypadku użycia, następnie poprzez przygotowanie zapytań sparql (modyfikujących OWL lub pobierających dane), kończąc na implementacji aplikacji klienckiej.

W kwestii reprezentacji danych, wykorzystanie takiego frameworku jak Apache Jena pozwala na intuicyjne operowanie na dowolnej reprezentacji ontologii, ich rozbudowę w srodowisku programowania java, oraz ich graficzną reprezentację. Wykorzystanie tutaj w pracy srodowiska java jednoczesnie czyni znacznie prostszą integrację tego rozwiązania z opisanym wczesniej mechanizmem transferowania informacji, który również napisany jest w tym języku.