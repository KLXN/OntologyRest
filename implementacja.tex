\section{Implementacja rozwiązania}
\label{cha:impl}

Rozwiązanie zaprezentowane w niniejszym artykule zostało napisane w języku Java, z wykorzystaniem Ontology API biblioteki Apache Jena. Biblioteka ta pozwala na mapowanie różnego rodzaju reprezentacji ontologii na klasy Java. Do transferowania naszych danych wykorzystamy bibliotekę Apache CXF, a dokładniej jej moduł REST, który pozwoli nam na przesyłanie zserializowanych danych przez protokół HTTP.

\subsection{Reprezentacja danych}
\label{sec:persist}

Ontology API biblioteki Apache Jena pozwala między innymi na budowanie oraz importowanie modeli ontologii. Stworzenie nowego modelu ontologii oraz zaimportowanie do niego już istniejącej ontologii zapisanej w formacie OWL przedstawiono na rysunku.

\subsection{Transfer danych}
\label{sec:transfer}

W projekcie uruchomiony został serwer HTTP Fuseki. Pozwala on na przechowywanie i udostepnienie danych wczytanych z plikow OWL poprzez zapytania HTTP. W jego wewnętrznej bazie danych budowane są grafu do których aplikacje klienckie odwołują się w zapytania SPARQL.

\subsubsection{Instalacja}

~Instalacja Jeny i Fuseki ogranicza się do dodania do zmiennych środowiskowych ścieżek do paczek.
\begin{lstlisting}
export JENA= /Users/XXXX/Fuseki/apache-jena-2.11.0/
export FUSEKI= /Users/XXXX/Fuseki/jena-fuseki-1.0.0/
\end{lstlisting}

\subsubsection{Uruchomienie serwera}

~Standardowo serwer FUSEKI uruchamiany jest pod wirtualną domeną " localhost " i na porcie 3030.

Uruchomienie serwera FUSEKI:
\begin{lstlisting}
./fuseki-server --update --loc=/Users/~/MyTDB/ /ds &
\end{lstlisting}

\subsubsection{Załadowanie pliku z ontologią}

~Załadowania pliku OWL (poprzez konsolową komendę):

\begin{lstlisting}
./s-put http://localhost:3030/ds/data default /Users/XXXX/RIMV3OWL.owl
\end{lstlisting}

\subsubsection{Tworzenie zapytania SPARQL}

~Wysłanie zapytania SPARQL wypisującego "koncepty" z załadowanego grafu:

\begin{enumerate}
\item Tworzenie zapytania.
\begin{lstlisting}
select distinct ?Concept where {[] a ?Concept}
\end{lstlisting}
\item Tworzenie zapytania HTTP POST, gdzie:
- jako URI podajemy sufix prowadzący do punktu końcowego (eng. endpoint) i doklajemy końcówkę /query
- w parametrze POST/GET "query" podajemy stworzone zapytanie 
- opcjonalne parametry: output (np. "json"), default-graph-uri
\item W efekcie otrzymujemy zapytanie (dla punktu końcowego "ds":
\end{enumerate}

\begin{lstlisting}
http://localhost:3030/ds/sparql? query=select+distinct++%3FConcept +where+%7B%5B%5D+a+%3FConcept%7D &default-graph-uri= &output=json
\end{lstlisting}