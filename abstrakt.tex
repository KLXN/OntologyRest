\section{Abstrakt}
\label{sec:Abstrakt}

Systemy medyczne należą do jednych z najbardziej rozbudowanych systemów w branży IT. Cechują się one nie tylko ogromnymi ilosciami danych, które przetwarzają, ale również bardzo dużą złożonoscią budowych tych danych oraz relacji, które między nimi zachodzą. Własnie branża e-health jako jedna z pierwszych zainteresowała się zastosowaniem ontologii oraz ich reprezentacji w zagadnieniu reprezentowania danych. Niniejszy artykuł porusza problem zastosowania optymalnych technologii i metodologii do implementacji systemu opartego o ontologię z naciskiem na przechowywanie oraz transfer ontologii.

Zaproponowane przez nas rozwiązanie problemu oparte zostało na języku programowania java oraz opartych na nim technologiach i narzędziach. Do transferu danych wykorzystano FUSEKI, które oferuje transfer danych protokołem REST z wykorzystaniem języka zapytań SPARQL. Dodatkowo przy definiowanu ontologii zaproponowano wykorzystanie opartego i interfejsy i stałe słownika danych.

Zaproponowana przez nas implementacja reprezentacji oraz transferu danych medycznych posiada wiele cech pożądanych przy projektowaniu tego typu systemów. Rozwiązanie to cechuje się bardzo dużą skalowalnoscią zarówno od strony wielkosci samego systemu jak i możliwosci rozbudowy przechowywanych w nim danych. Zastosowanie architektury typu SOA przy budowie systemu medyczneo pozwala na łatwą, modułową rozbudowę takiego systemu oraz ułatwioną integrację z innymi systemami medycznymi, zupełnie niezależnymi od siebie.

