\section{Wstep}
\label{cha:wstep}

\subsection{Ontologia}
\label{sec:ont}

\subsection{OWL}
\label{sec:owl}

OWL (Ontology Web Language) jest opartym na składni XML językiem służącym do opisu ontologii. Stanowi on rozszerzenie języka RDF (Resource Definition Language). Istnieją 3 odmiany OWL: OWL Lite, OWL DL oraz OWL Full. W 2004 roku język ten został uznany za standard przez organizację W3C.

OWL został stworzony do przetwarzania informacji o obiektach oraz relacji między nimi, nie do przechowywania tych informacji w formie czytelnej dla człowieka. OWL jest zbudowany na bazie RDF, tak więc języki te są ze sobą w pełni kompatybilne. OWL posiada jednak znacznie mocniejszą składnię.

Do elementów języka OWL należą takie atrybuty jak:

\begin{itemize}

\item Class - definiuje grupę indywidualnosci, które posiadają pewne wspólne cechy. Klasy można organizować w hierarchie za pomocą atrybutu subClassOf.


\item rdf:Property - okresla pewne relacje między indywidualnosciami np. samochod może posiadać drzwi, osoba moze posiadać dziecko albo rodzica.

\item rdfs:domain - ogranicza indywidualności, do których można zastosować relację (property). Jeżeli relacja łączy jedną indywidualność z drugą, i jednocześnie posiada ona domenę na określoną klasę, to obie indywidualności muszą należeć do tej klasy.

\item rdfs:range - ogranicza nie tylko indywidualności, ale także wartość relacji (property)

\item Individual - indywidualności są instancjami klas, które są połączone pewnymi relacjami (property).

\end{itemize}

Wszystkie powyższe atrybuty są częścią specyfikacji OWL Lite.



\subsection{SPARQL}
\label{sec:sparql}

\subsection{REST}
\label{sec:rest}

REST jest protokołem komunikacji między aplikacjami, który tak naprawdę jest tylko pewną dodatkową warstwą nałożoną na protokół HTTP. Budowanie zapytań restowych z poziomu języków programowania jest tak samo proste, jak zrozumienie działania protokołu HTTP. Istnieje wiele bibliotek i prostych frameworków do budowania serwisów, odbierających i wysyłających zapytania HTTP. Należą do nich np Jersey-RS, czy zbudowana na jego bazie trochę większa biblioteka Apache CXF. Biblioteka Fuserki, będąca częścią specyfikacji Apache Jena oferuje możliwość transferu danych właśnie protokołem REST, przy użyciu języka zapytań SPAQRL. 

Ze względu na swoją prostotę i oparcie o standard HTT, wybór tego protokołu jako sposobu komunikacji między aplikacjami jest bardzo dobrą decyzją. Umożliwia to znakomitą skalowalność naszego systemu, poprzez rozdzielenie go na szereg indywidualnych aplikacji, które nie muszą być napisane przy użyciu tej samej technologii, muszą jedynie współdzielić tą metodę komunikacji oraz ewentualnie sposób serializacji danych.

