\section{Wstep}
\label{cha:wstep}

\subsection{Ontologia}
\label{sec:ont}

Ontologia jest formalną reprezentacją pewnej dziedziny wiedzy. Składają się na nią zapis zbiorów pojęć i relacji między nimi. Zapis ten tworzy schemat pojęciowy, który jest opisem danej dziedziny wiedzy. Może służyć jednocześnie jako podstawa do wnioskowania o właściwości opisywanych ontologią pojęć.	

Pod pojęciem ontologii mogą się kryć różne struktury wiedzy. Ich przeznaczenie czy zakres stosowania może być wieloraki.

Ontologie możemy podzielić ze względu na stopień ich formalizacji:
\begin{itemize}
\item nieformalne
	\begin{itemize}
		\item predefiniowane słownictow
		\item słowniki
		\item tezarusy
		\item taksonomie
	\end{itemize}
\item formalne
	\begin{itemize}
		\item ontologie oparte na danych
		\item ontologie oparte na logice
	\end{itemize}
\end{itemize}

Podział następuje także ze względu na zakres stosowania:
\begin{itemize}
\item ontologie wysokiego poziomu (\textit{upper ontologies})
\item ontologie dziedzinowe (\textit{domain ontologies})
\item ontologie aplikacyjne
\end{itemize}

Istnieje szereg języków pozwalających na zapis lub wspierających ontologie:
\begin{itemize}
\item OWL (\textit{Ontology Web Language})
\item RDF (\textit{Resource Description Framework})
\item RDF Vocabulary Description Language (RDF Schema, RDFS)
\item OCML (Operational Conceptual Modeling Language)
\item XML (Extensible Markup Language)
\end{itemize}

\subsection{OWL}
\label{sec:owl}

OWL (Ontology Web Language) jest opartym na składni XML językiem służącym do opisu ontologii. Stanowi on rozszerzenie języka RDF (Resource Definition Language). Istnieją 3 odmiany OWL: OWL Lite, OWL DL oraz OWL Full. W 2004 roku język ten został uznany za standard przez organizację W3C.

OWL został stworzony do przetwarzania informacji o obiektach oraz relacji między nimi, nie do przechowywania tych informacji w formie czytelnej dla człowieka. OWL jest zbudowany na bazie RDF, tak więc języki te są ze sobą w pełni kompatybilne. OWL posiada jednak znacznie mocniejszą składnię.

Do elementów języka OWL należą takie atrybuty jak:

\begin{itemize}

\item Class - definiuje grupę indywidualnosci, które posiadają pewne wspólne cechy. Klasy można organizować w hierarchie za pomocą atrybutu subClassOf.


\item rdf:Property - okresla pewne relacje między indywidualnosciami np. samochod może posiadać drzwi, osoba moze posiadać dziecko albo rodzica.

\item rdfs:domain - ogranicza indywidualności, do których można zastosować relację (property). Jeżeli relacja łączy jedną indywidualność z drugą, i jednocześnie posiada ona domenę na określoną klasę, to obie indywidualności muszą należeć do tej klasy.

\item rdfs:range - ogranicza nie tylko indywidualności, ale także wartość relacji (property)

\item Individual - indywidualności są instancjami klas, które są połączone pewnymi relacjami (property).

\end{itemize}

Wszystkie powyższe atrybuty są częścią specyfikacji OWL Lite.

\subsection{Protege}
\label{sec:protege}
Protege jest otwarto-źródłowym narzędziem stworzonym przez Stanford Medical Informatics. Jak przy większości narzędzi modelujących, architektura Protege jest przejrzyście podzielona na dwie części - model i widok. Modele są wewnętrzną reprezentacją mechanizmy ontologii i baz wiedzy. Widoki zapewniają interfejs użytkownika, dzięki któremu można wyświetlać i manipulować modelami.

Protege posiada możliwość wczytania, edycji i zapisywania w wielu popularnych dla ontologii formatach, np:
\begin{itemize}
\item RDF
\item OWL
\item XML
\item UML
\item Bazy relacyjne
\end{itemize}

\subsection{SPARQL}
\label{sec:sparql}

\subsection{REST}
\label{sec:rest}

REST jest protokołem komunikacji między aplikacjami, który tak naprawdę jest tylko pewną dodatkową warstwą nałożoną na protokół HTTP. Budowanie zapytań restowych z poziomu języków programowania jest tak samo proste, jak zrozumienie działania protokołu HTTP. Istnieje wiele bibliotek i prostych frameworków do budowania serwisów, odbierających i wysyłających zapytania HTTP. Należą do nich np Jersey-RS, czy zbudowana na jego bazie trochę większa biblioteka Apache CXF. Biblioteka Fuserki, będąca częścią specyfikacji Apache Jena oferuje możliwość transferu danych właśnie protokołem REST, przy użyciu języka zapytań SPAQRL. 

Ze względu na swoją prostotę i oparcie o standard HTTP, wybór tego protokołu jako sposobu komunikacji między aplikacjami jest bardzo dobrą decyzją. Umożliwia to znakomitą skalowalność naszego systemu, poprzez rozdzielenie go na szereg indywidualnych aplikacji, które nie muszą być napisane przy użyciu tej samej technologii, muszą jedynie współdzielić tą metodę komunikacji oraz ewentualnie sposób serializacji danych.

