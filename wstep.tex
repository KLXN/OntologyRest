\section{Wstęp}
\label{cha:wstep}

Dane medyczne są podstawowym zbiorem informacji odności pacjenta, historii jego choroby oraz planowanego leczenia. W 40 milinonowym kraju, takim jakim jest Polska, liczba tych danych rośnie w zatrważającym tempie. Coraz to nowe firmy prześcigają się w podawaniu gotowych rozwiązań dotyczących przetwarzania i przechowywania tychże danych. Problem ten dotyka nie tylko naszego kraju ale niemal wszystkich wysokorozwiniętych nacji świata.

Jest to problem złożony. Możemy podzielić go na kilka części:
\begin{itemize}
	\item gromadzenie danych
	\item aktualizowanie zgromadzonych danych
	\item przekazywanie zgromadzonych i aktualizowanych danych
	\item ochrona danych
	\item wprowadzanie nowych podmiotów mających dostęp do zgromadzonych danych
	\item utrata i odzyskiwanie danych
	\item a wreszcie utylizacja danych
\end{itemize}

Poszególne problemy pokrótce przeanalizujemy na podstawie polskiego systemu służby zdrowia co pozwoli nam łatwiej zrozumieć działalność systemu HL7, który obecnie jest jednym z popularniejszych na świecie. \cite{20} Przypomnijmy iż w Polsce używa się obecnie systemu eWUŚ (Elektroniczna Weryfikacja Uprawnień Świadczeniobiorców) oraz zintegrowanego z nim systemu ZIP (Zintegrowany Informator Pacjenta).

Problem z gromadzeniem danych można uznać za problem najważniejszy gdyż istotną w nim rolę odgrywa czynnik ludzki. Jedna pomyłka podczas wprowadzania informacji rzutuje na działalności całego systemu. \cite{25} Przykład: Błędnie wprowadzone nazwisko podczas rejestracji pacjenta pociągnie za sobą konsekwencje w postaci nie odnalezienia go wśród pacjentów ubezpieczonych oraz automatyczne generowanie faktury za udzielone świadczenia.

Aktualizowanie danych sugeruje nam, a wręcz niejako wymusza aby nasza baza danych była skalowalna. \cite{21} A co za tym idzie musi posiadać określoną pojemność, która będzie rozszerzalna. Dlatego przechowywanie takich danych musi zostać rozproszone w celach zarówno wydajności działania jaki i bezpieczeństwa.
Skoro jesteśmy już przy bezpieczeństwie, warto wspomnieć o ochronie danych. Wspomniane przez nas systemy są chronione w sposób prosty. Dostęp do nich mają okreśne podmioty zarejestrowane w konkretnej bazie, posługujące się przypisanymi tylko im loginami i hasłami. \cite{25} Przykład: Weryfikacja uprawnień świadczeniobiorcy odbywa się poprzez wpisanie jego numeru ewidencyjnego PESEL do wyszukiwarki bazy danych. Osoba, która ten numer wpisuje musi być uprzednio zalogowna do systemu a przed każdą próbą skorzystania z niego, potwierdzać swoją tożsamość.

W ten oto sposób pokrótce opisaliśmy wybrane problemy dotyczące przechowywania i przesyłania informacji. Ninejszym wskazaliśmy, w którą stronę podążają wspomniane problemy. W kolejnych rozdziałach wspróbujemy zastanowić się wspólnie nad uszeregowaniem zarówno kłopotów jak i przypisywanych im rozwiązań, które pojawiają się w temacie systemów przechowywania i przesyłania danych medycznych.